%  LaTeX template for abstract submission for MCM 2019
% 
%  If this paper is for a special (invited) session, put the name of session organizer 
%  (first name, family name), and the session title.
%  For plenary talks, put ``Plenary Talk'' for the session title, and Zdravko Botev for the organizer.
%
\insession{Peter}{Kritzer}{Current Challenges in High-Dimensional Algorithms}

% First name and name of the speaker.
\speaker{Yuhan}{Ding}%
%  (put no space here)
% Title of the talk, capitalized.
\title{An Optimal Adaptive Algorithm Based on the Decay Rate of the Series Coefficients of the Input Fuctions }

% For each author, give the first name, family name, affiliation, and email.
% Ideally, the affiliation and email should fit on a single line.  
% No need to put the full snail mailing address.  
\author{Yuhan}{Ding}{Misericordia University, USA}{yding@misericordia.edu}
\author{Fred}{Hickernell}{Illinois Institute of Technology, USA}{hickernell@iit.edu}
\author{Peter}{Kritzer}{Austrian Academy of Sciences, Austria}{peter.kritzer@oeaw.ac.at}
\author{Simon}{Mak}{Georgia Institute of Technology, USA}{smak6@gatech.edu}


% Type your abstract here.
\abstract{Automatic algorithms attempt to provide approximate solutions that differ from exact solutions by no more than a user-specified error tolerance. This talk describes an automatic adaptive algorithm by tracking the decay rate of the series coefficients of the input functions.  We assume that the Fourier coefficients of the input function to be approximated decay sufficiently fast, but do not require the decay rate to be known a priori. We also assume that the Fourier coefficients decay steadily, although not necessarily monotonically.   Under these assumptions, our adaptive algorithm is shown to achieve an approximation to the function satisfying the desired error tolerance, without prior knowledge of the norm of the function to be approximated.  Moreover, the computational cost of our algorithm and the lower complexity bound of this kind of problem are presented in the talk. With this, our algorithm is shown to be essentially optimal.  In addition to that,  the tractablity of this kind of problem is discussed.

	
	
%  If you have refererences, put them here in a format like below. 
%  This can be obtained using BiBTeX with the bib style plain.bst. 
%  Note that this must be placed inside the abstract.
%\begin{thebibliography}{1}

%\bibitem{kroese2011handbook}
%D. P. Kroese, T. Taimre, and Z. I. Botev.
%\newblock {\em Handbook of Monte Carlo methods}.
%\newblock John Wiley \& Sons, 2011.


%\bibitem{owen2018monte}
%A. B. Owen  and P. W. Glynn.
%\newblock {\em Monte Carlo and Quasi-Monte Carlo Methods: MCQMC 2016, Stanford, CA, August 14-19}.
%\newblock Springer, Volume 241, 2018.


%\end{thebibliography}
}  % End of abstract.


